% Options for packages loaded elsewhere
% Options for packages loaded elsewhere
\PassOptionsToPackage{unicode}{hyperref}
\PassOptionsToPackage{hyphens}{url}
\PassOptionsToPackage{dvipsnames,svgnames,x11names}{xcolor}
%
\documentclass[
  british,
  a4paper,
]{article}
\usepackage{xcolor}
\usepackage{amsmath,amssymb}
\setcounter{secnumdepth}{5}
\usepackage{iftex}
\ifPDFTeX
  \usepackage[T1]{fontenc}
  \usepackage[utf8]{inputenc}
  \usepackage{textcomp} % provide euro and other symbols
\else % if luatex or xetex
  \usepackage{unicode-math} % this also loads fontspec
  \defaultfontfeatures{Scale=MatchLowercase}
  \defaultfontfeatures[\rmfamily]{Ligatures=TeX,Scale=1}
\fi
\usepackage{lmodern}
\ifPDFTeX\else
  % xetex/luatex font selection
\fi
% Use upquote if available, for straight quotes in verbatim environments
\IfFileExists{upquote.sty}{\usepackage{upquote}}{}
\IfFileExists{microtype.sty}{% use microtype if available
  \usepackage[]{microtype}
  \UseMicrotypeSet[protrusion]{basicmath} % disable protrusion for tt fonts
}{}
\makeatletter
\@ifundefined{KOMAClassName}{% if non-KOMA class
  \IfFileExists{parskip.sty}{%
    \usepackage{parskip}
  }{% else
    \setlength{\parindent}{0pt}
    \setlength{\parskip}{6pt plus 2pt minus 1pt}}
}{% if KOMA class
  \KOMAoptions{parskip=half}}
\makeatother
% Make \paragraph and \subparagraph free-standing
\makeatletter
\ifx\paragraph\undefined\else
  \let\oldparagraph\paragraph
  \renewcommand{\paragraph}{
    \@ifstar
      \xxxParagraphStar
      \xxxParagraphNoStar
  }
  \newcommand{\xxxParagraphStar}[1]{\oldparagraph*{#1}\mbox{}}
  \newcommand{\xxxParagraphNoStar}[1]{\oldparagraph{#1}\mbox{}}
\fi
\ifx\subparagraph\undefined\else
  \let\oldsubparagraph\subparagraph
  \renewcommand{\subparagraph}{
    \@ifstar
      \xxxSubParagraphStar
      \xxxSubParagraphNoStar
  }
  \newcommand{\xxxSubParagraphStar}[1]{\oldsubparagraph*{#1}\mbox{}}
  \newcommand{\xxxSubParagraphNoStar}[1]{\oldsubparagraph{#1}\mbox{}}
\fi
\makeatother


\usepackage{longtable,booktabs,array}
\usepackage{calc} % for calculating minipage widths
% Correct order of tables after \paragraph or \subparagraph
\usepackage{etoolbox}
\makeatletter
\patchcmd\longtable{\par}{\if@noskipsec\mbox{}\fi\par}{}{}
\makeatother
% Allow footnotes in longtable head/foot
\IfFileExists{footnotehyper.sty}{\usepackage{footnotehyper}}{\usepackage{footnote}}
\makesavenoteenv{longtable}
\usepackage{graphicx}
\makeatletter
\newsavebox\pandoc@box
\newcommand*\pandocbounded[1]{% scales image to fit in text height/width
  \sbox\pandoc@box{#1}%
  \Gscale@div\@tempa{\textheight}{\dimexpr\ht\pandoc@box+\dp\pandoc@box\relax}%
  \Gscale@div\@tempb{\linewidth}{\wd\pandoc@box}%
  \ifdim\@tempb\p@<\@tempa\p@\let\@tempa\@tempb\fi% select the smaller of both
  \ifdim\@tempa\p@<\p@\scalebox{\@tempa}{\usebox\pandoc@box}%
  \else\usebox{\pandoc@box}%
  \fi%
}
% Set default figure placement to htbp
\def\fps@figure{htbp}
\makeatother


% definitions for citeproc citations
\NewDocumentCommand\citeproctext{}{}
\NewDocumentCommand\citeproc{mm}{%
  \begingroup\def\citeproctext{#2}\cite{#1}\endgroup}
\makeatletter
 % allow citations to break across lines
 \let\@cite@ofmt\@firstofone
 % avoid brackets around text for \cite:
 \def\@biblabel#1{}
 \def\@cite#1#2{{#1\if@tempswa , #2\fi}}
\makeatother
\newlength{\cslhangindent}
\setlength{\cslhangindent}{1.5em}
\newlength{\csllabelwidth}
\setlength{\csllabelwidth}{3em}
\newenvironment{CSLReferences}[2] % #1 hanging-indent, #2 entry-spacing
 {\begin{list}{}{%
  \setlength{\itemindent}{0pt}
  \setlength{\leftmargin}{0pt}
  \setlength{\parsep}{0pt}
  % turn on hanging indent if param 1 is 1
  \ifodd #1
   \setlength{\leftmargin}{\cslhangindent}
   \setlength{\itemindent}{-1\cslhangindent}
  \fi
  % set entry spacing
  \setlength{\itemsep}{#2\baselineskip}}}
 {\end{list}}
\usepackage{calc}
\newcommand{\CSLBlock}[1]{\hfill\break\parbox[t]{\linewidth}{\strut\ignorespaces#1\strut}}
\newcommand{\CSLLeftMargin}[1]{\parbox[t]{\csllabelwidth}{\strut#1\strut}}
\newcommand{\CSLRightInline}[1]{\parbox[t]{\linewidth - \csllabelwidth}{\strut#1\strut}}
\newcommand{\CSLIndent}[1]{\hspace{\cslhangindent}#1}

\ifLuaTeX
\usepackage[bidi=basic]{babel}
\else
\usepackage[bidi=default]{babel}
\fi
% get rid of language-specific shorthands (see #6817):
\let\LanguageShortHands\languageshorthands
\def\languageshorthands#1{}
\ifLuaTeX
  \usepackage[english]{selnolig} % disable illegal ligatures
\fi


\setlength{\emergencystretch}{3em} % prevent overfull lines

\providecommand{\tightlist}{%
  \setlength{\itemsep}{0pt}\setlength{\parskip}{0pt}}



 


\makeatletter
\@ifpackageloaded{caption}{}{\usepackage{caption}}
\AtBeginDocument{%
\ifdefined\contentsname
  \renewcommand*\contentsname{Table of contents}
\else
  \newcommand\contentsname{Table of contents}
\fi
\ifdefined\listfigurename
  \renewcommand*\listfigurename{List of Figures}
\else
  \newcommand\listfigurename{List of Figures}
\fi
\ifdefined\listtablename
  \renewcommand*\listtablename{List of Tables}
\else
  \newcommand\listtablename{List of Tables}
\fi
\ifdefined\figurename
  \renewcommand*\figurename{Figure}
\else
  \newcommand\figurename{Figure}
\fi
\ifdefined\tablename
  \renewcommand*\tablename{Table}
\else
  \newcommand\tablename{Table}
\fi
}
\@ifpackageloaded{float}{}{\usepackage{float}}
\floatstyle{ruled}
\@ifundefined{c@chapter}{\newfloat{codelisting}{h}{lop}}{\newfloat{codelisting}{h}{lop}[chapter]}
\floatname{codelisting}{Listing}
\newcommand*\listoflistings{\listof{codelisting}{List of Listings}}
\makeatother
\makeatletter
\makeatother
\makeatletter
\@ifpackageloaded{caption}{}{\usepackage{caption}}
\@ifpackageloaded{subcaption}{}{\usepackage{subcaption}}
\makeatother
\usepackage{bookmark}
\IfFileExists{xurl.sty}{\usepackage{xurl}}{} % add URL line breaks if available
\urlstyle{same}
\hypersetup{
  pdftitle={Is reproducibility good enough?},
  pdfauthor={Harald Bjarne Vika},
  pdflang={en-GB},
  colorlinks=true,
  linkcolor={blue},
  filecolor={Maroon},
  citecolor={Blue},
  urlcolor={Blue},
  pdfcreator={LaTeX via pandoc}}


\title{Is reproducibility good enough?}
\author{Harald Bjarne Vika}
\date{18,09,2025}
\begin{document}
\maketitle
\begin{abstract}
A very short abstract. Put the abstract text here. One or two paragraphs
summarising what follows below.
\end{abstract}


\section{Introduction}\label{introduction}

What is this paper about? What is discussed? Why is it of any
consequence?

The advancement of science is build on a trust in discovery of new data,
and through the use of reproducibility, scientist can gain confidence in
their research concluded data. But in recent time, the community have
been getting reports on more peer-reviewed preclinical studies that are
none reproducible(McNutt, 2014).

In this paper we will look more in to reproducibility of scientific
data, why its important for using to gaining trust and if its good
enough to be consider useful for the future scientist to use. The paper
will first present literature around the meaning of word reproducibility
and replicate, and its use in the science world. We will then discuss if
reproducibility is necessary or if the scientific data can be
replicated. How the use of R and a quarto document can help with
reproducibility of research, any problems that can which may occur and
if we can solve them. In the need we will try to conclude these
questions and what our thoughts on the subject of reproducibility of
scientific data.

\section{Literature review}\label{literature-review}

\section{Theory on reproducibility}\label{theory-on-reproducibility}

Smart stuff from others about the topic.

The cover story of The Economist ``How Science goes wrong'' uses the
terms for reproducibility from (Barba, n.d.) which is
``\emph{Reproducibility refers to the instance in which the original
researcher's data and computer codes are used to regenerate the results,
and arriving at the same results using their own data and methods}''.
While from the same review paper the term of replicability can viewed as
an ``\emph{instances in which a resarcher collects new data to arrive at
the same scientific findings as a previous study,''} or \emph{''a
different team arriving at the same results using the original authors
artifacts''.} With this we have a understanding that the terms of
reproducibility and replicability can overlap one another but can be
used differently. Another term is generalizability, where the results of
a study can be applied on any other context or population which is
different from the original case study. These three terms would refers
to the science that is''robust and reliable'' which is the foundation of
all scientific development for the ability to build on prior work. Here
we will try to focus on the use of reproducibility of scientific data.

The reproducibility is a necessary, but not sufficient condition for
replicability. Peng (2011) Discusses that reproducibility for a
publication should be a minimal requirement. As said earlier that
reproducibility is a way for the scientist to gain confident and trust
in their research but what is a research reproducibility? Goodman et al.
(2016) gives us three explanations of what research reproducibiliy
means:

\begin{itemize}
\item
  Methods reproducibility, which is how the provision of enough detail
  about study procedures and data can be used to repeat the same
  research in theory and in actuality.
\item
  Results reproducibility, also described as replicabilty, is refered to
  obtain the same results from previously research when conducting
  independent studys but the process is the same as the original
  research.
\item
  Robustness and generalizability are two terms that can be used instead
  of the term reproducibility, where robustness means the stability of a
  experimental conclusion when it is either a baseline assumptions or
  experimental procedures. Generalizability can be refereed to the
  persistence of an effect in settings different from and outside of
  experimental framework.
\end{itemize}

\section{Publication bias}\label{publication-bias}

When it comes to publication on reproducibility one can encounter
publication bias which may affect the ability to publish the findings.
The worst case if a scientific journals contains a big collection of the
type 1 error. So what is type 1 error? Its the definition of a
incorrectly concluding of the research. (kilde) Its the size of the
error which the researcher is willing to accept prior to the hypothesis
test. Type 1 error is when H0 is rejected even if its true. It can be
about concluding that there is an effect even when its not. This way we
may end up with similar studies with no effect but there can be a risk
that we would find the wrong conclusion in the literature. This will
hurt later research that tries to replicate the experiment. An exempel
on this is the ``File Drawer Problem'' from Rosenthal (1979), where
studies with an H0 that can not be rejectet is sent to the file drawer
and the ones where H0 is rejected is published. This will go on to
create false positive. If prestigious journals manage to publish an
research with a false positive it will make it so that others will have
little motivation to try to reproduce it. Another error that can occur
is that its costly to invest in a research program which in the end will
create more false positive and may lead to ineffective policy changes. A
field that is known to publish research with false positive would surly
risk losing its credibility and trust from other scientist (Simmons et
al., 2011).

Use a least 20 citations, a least 5 of them must be new (not from the
provided .bib file).

Use both in-line and normal citations.

Example:

(\textbf{gentleman2005?}) argues that bla bla bla. On the other hand
it's claimed that bla bla (\textbf{barbalorenaa.2018?};
\textbf{bartlett2008?}).

\section{Discussion of the reseach
question}\label{discussion-of-the-reseach-question}

\begin{itemize}
\tightlist
\item
  Should replicability be the norm or is this to much to ask for now?
\item
  Can Quarto documents help with reproducibility?
\item
  What problems remains and how can these be solved?
\end{itemize}

\section{Conclusion}\label{conclusion}

\section{References}\label{references}

and

\begin{itemize}
\tightlist
\item
  Version number and reference to packages used
\item
  R version used
\end{itemize}

\phantomsection\label{refs}
\begin{CSLReferences}{1}{0}
\bibitem[\citeproctext]{ref-barbaTerminologiesReproducibleResearch2018}
Barba, L. A. (n.d.). \emph{Terminologies for reproducible research}.
\url{https://doi.org/10.48550/arXiv.1802.03311}

\bibitem[\citeproctext]{ref-goodman2016b}
Goodman, S. N., Fanelli, D., \& Ioannidis, J. P. A. (2016). What does
research reproducibility mean? \emph{Science Translational Medicine},
\emph{8}(341), 341ps12--341ps12.

\bibitem[\citeproctext]{ref-mcnutt2014}
McNutt, M. (2014). Reproducibility. \emph{Science}, \emph{343}(6168),
229--229.

\bibitem[\citeproctext]{ref-peng2011}
Peng, R. D. (2011).
\href{https://www.ncbi.nlm.nih.gov/pubmed/22144613}{Reproducible
{Research} in {Computational Science}}. \emph{Science},
\emph{334}(6060), 1226--1227.

\bibitem[\citeproctext]{ref-rosenthal1979}
Rosenthal, R. (1979). \emph{The file drawer problem and tolerance for
null results.}

\bibitem[\citeproctext]{ref-simmons2011}
Simmons, J. P., Nelson, L. D., \& Simonsohn, U. (2011). False-positive
psychology: {Undisclosed} flexibility in data collection and analysis
allows presenting anything as significant. \emph{Psychological Science},
\emph{22}(11), 1359--1366.

\end{CSLReferences}




\end{document}
